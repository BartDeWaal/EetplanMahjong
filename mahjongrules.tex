\documentclass{article}

% This type of document will contain a lot of lists and short sentences, so we want no indented paragraphs and tight lists
\usepackage{parskip}
\usepackage{mdwlist}

\begin{document}
% I tried to add Chinese, but I failed miserably
\title{Eetplan Rules of Mahjong}
\author{Bart}
\maketitle

\section{General Considerations}
This document is meant to describe the way we play mahjong at eetplan. The goal of the game is to make either money or "money". The real goal of the game is to learn chinese.

Rules may be exchanged for logically equivalent ones if customary. For instance, normally the first player will take his first tile while taking his 13 tiles. Taking 14 tiles like this is logically the same as taking 13 and then 1, so this is fine. I've decided not to describe the complete wall procedure, as it's not required for the game.
% TODO: describe the procedure.

Here we describe Mahjong as played by four players. If a different number is wanted, they will have to agree on the rules.

The game is played by playing rounds followed by rounds, until the players decide to stop.

To begin the game, give all players the same amount of points and choose a first host.

% TODO: describe what equipment we use
\section{Definitions}
\emph{The wall} is the set of all tiles that would normally be drawn from, normally arranged in a pleasing manner.

The \emph{Kong Stack} is 12 tiles kept seperate from the rest of the wall. It's used to top up a players hand when they get a Kong.

Two tiles are \emph{equivalent} if the faces contain the same information. So if either of the tiles are from a suit, they need to have both the same suit and the same number on them.

A \emph{pair} is two equivalent tiles.

A \emph{set} is any one of:
\begin{enumerate*}
    \item A \emph{Pung}, three of the same tile,
    \item A \emph{Kong}, four of the same tile, or
    \item A \emph{Chow}, a run of three tiles in the same suit.
\end{enumerate*}

A player \emph{completes a hand} if it's her turn and she has four sets and a pair.

\section{Playing a round}
At the start of each round the tiles are shuffled and arranged into a wall. Each player takes 13 random tiles from the wall. 12 tiles are kept seperate from the wall as the "kong stack".

The host starts his turn by taking a new tile.

A turn works as follows:
\begin{enumerate*}
    \item A player starts the turn by one of the following options. If multiple people can start the turn, priority goes to the highest item on the list. If multiple people can complete their hand, priority goes to the person who is closest in counterclockwise direction.
    \begin{enumerate*}
        \item A player is the host during the first turn.
        \item A player completes their hand using the tile that was just discarded.
        \item A player makes a pung using the tile that was just discarded. Any player that can make a pung using that tile can do this.

        First, she shouts "Pung". She takes the tiles, and puts it open in front of her hand. Then she takes two of the same tile from her hand and adds them, open, next to the tile she just took.
        \item A player makes a Kong using the tile that was just discarded. Any player that can make a kong using that tile and three hidden tiles in her hand can do this.

        First, she shouts "Kong". She takes the tiles, and puts it open in front of her hand. Then she takes three of the same tile that were hidden in her hand and adds them, open, next to the tile she just took.

        Now she takes a tile from the Kong stack to replace the fourth tile she used for that set.
        \item A player takes a new tile from the wall. You can only do this if you are on the right of the last person to have a turn.
    \end{enumerate*}
    \item If a player has a pung hidden in their hand she can declare it and seperate the four tiles, face down, from the rest of her hand. She takes a new tile from the pung stack.
    \item If a player has a hidden tile in their hand that matches a revealed kong she has, she can turn the kong into a pung using that tile. She takes a new tile from the pung stack.
    \item The player ends the turn:
   a. If the player has completed her hand, she says this. She does not reveal her hand until the end of the round.
   b. Otherwise, she discards a tile from her hand.
\end{enumerate*}

A round ends when all players have completed their hand, or the wall has no more tiles left.

\subsection{Scoring a round}
Points are exchanged according to the following rules:
\begin{itemize*}
    \item If player A completed her hand before player B, then player A never has to give points to player B
    \item If a player gives or receives points and she is host, she receives double points.
\end{itemize*}
\end{document}
